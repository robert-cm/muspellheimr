\documentclass{article}
\title{SFMLRPG}
\description{Roguelike game written in C++ using SFML 2.0}
\author{Robert Muschong}
\coordonator{Cosmin Bonchis}
\date{October 2014}
\begin{document}
	\maketitle
	Muspellsheimr
\\
\section{Introduction}
	\subsection{SFML}
\\
SFML. What is SFML? How is it useful?
\\
SFML provides a simple interface to the various components of your PC, to ease
the development of games and multimedia applications. It is composed of five 
modules: system, window, graphics, audio and network.
\\
With SFML, your application can compile and run out of the box on the most common
operating systems: Windows, Linux, Mac OS X and soon Android & iOS. Pre-compiled 
SDKs for your favorite OS are available on the download page.
\\
SFML has official bindings for the C and .Net languages. And thanks to its active
community, it is also available in many other languages such as Java, Ruby, Python,
Go, and more. Learn more about them on the bindings page.
\\
	\subsection{Roguelite}
\\
Roguelike is a sub-genre of role-playing video games, characterized by procedural
level generation, turn-based gameplay, tile-based graphics and permanent death,
and typically based on a high fantasy narrative setting. Roguelikes descend from
the 1980 game Rogue, particularly mirroring Rogue's character- or sprite-based
graphics, turn-based gameplay that gives the player the time to plan each move,
and high fantasy setting. In more recent years, new variations of roguelikes
incorporating other gameplay genres, thematic elements and graphical styles have
become popular, and are sometimes called roguelike-like, rogue-lite or
procedural death labyrinths to reflect the variation from titles which mimic
the gameplay of traditional roguelikes more faithfully.
\\
A statistic (or stat) in role-playing games is a piece of data that represents
a particular aspect of a fictional character. That piece of data is usually
a (unitless) integer or, in some cases, a set of dice. For some types of statistics,
this value may be accompanied with a descriptive adjective, sometimes called a
specialisation or aspect, that either describes how the character developed that
particular score or an affinity for a particular use of that statistic.
Most games divide their statistics into several categories. The set of categories
actually used in a game system, as well as the precise statistics within each
category, vary greatly.
The most often used types of statistic include:
\\
Attributes describe to what extent a character possesses natural, in-born
characteristics common to all characters.
Advantages and disadvantages are useful or problematic characteristics that are
not common to all characters.
Powers represent unique or special qualities of the character. In game terms,
these often grant the character the potential to gain or develop certain advantages
or to learn and use certain skills.
Skills represent a character's learned abilities in predefined areas.
Traits are broad areas of expertise, similar to skills, but with a broader and
usually more loosely defined scope, in areas freely chosen by the player.
\\
\section{Planning}
	1. Engine implementation using SFML(2.{0?})
	2. A* or Djikstra implementation		//TODO: decide which one to use
	3. Procedural map generator
	4. Basic groundwork for the project
	5. Dice roll implementation for Player class	//Enemy class inherits Player
	6. Code refactoring && pre 1.0 release
	7. 1.0 Release
\\
\section{Conclusion}
\\
//TODO: to be added at a later time /*Lorem ipsum filler*/
Lorem ipsum dolor sit amet, consectetur adipisicing elit, sed do eiusmod tempor
incididunt ut labore et dolore magna aliqua. Ut enim ad minim veniam, quis
nostrud exercitation ullamco laboris nisi ut aliquip ex ea commodo consequat.
Duis aute irure dolor in reprehenderit in voluptate velit esse cillum dolore eu
fugiat nulla pariatur.
\\
\end{document}
